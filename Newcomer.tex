
\paragraph{Definition:} Unless there is a new person to talk to, a lot of
the ``education stuff'' we do could grow pretty stale. Many of the
patterns and use cases for peeragogy assume that there will be an
audience or a new generation of learners.

\paragraph{Problem:} Some of the problems are well summed up with a quote:

\begin{quote}
\textbf{Régis Barondeau}: I joined this handbook project late, making me
a ``newcomer''. When I started to catch up, I rapidly faced doubts:
Where do I start? How can I help? How will I make it, having to read
more than 700 posts to catch up? What tools are we using ? How do I use
them? Etc. Although this project is amazingly interesting, catching the
train while it already reached high speed can be an extreme sport. By
taking care of newcomers, we might avoid losing valuable contributors
because they don't know how and where to start, and keep our own project
on track.
\end{quote}

\paragraph{Solution:} It is good to try to become aware of what a newcomer
needs, and what their motivations are. Another quote can illustrate:

\begin{quote}
\textbf{Charlotte Pierce}: Joe was working a lot on the book, and I
thought ``this is interesting hard work, and he shouldn't have to do
this alone.'' As a Peeragogy newcomer, I was kindly welcomed and
mentored by Joe, Howard, Fabrizio, and others. I asked naive questions
and was met with patient answers, guiding questions, and resource links.
Concurrently, I bootstrapped myself into a position to contribute to the
workflow by editing the live manuscript for consistency, style, and
continuity.
\end{quote}

\paragraph{Challenges:} Newcomers in the Peeragogy project have often
complained about feeling confused, suggesting that our project roadmap
that the newcomer and
the \href{http://peeragogy.org/practice/roadmap/}{roadmap} may not be
sufficiently clear, and that more work has to be done the project
accessible. Even in the absence of actual newcomers, we need to try and
look at things with a beginner's mind.

\paragraph{What's Next:} We recently revised the ``How to Get Involved''
page, listing the top ten sites we use. Another reasonable thing to post
would be a top-ten list of activities, so that people can get an easier
view on the kinds of things we do in the project.
