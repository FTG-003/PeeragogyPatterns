\section{Newcomer}\label{sec:Newcomer}
\subsubsection*{Context}
A lot of ``education'' assumes we are speaking to a new generation. 
In learning more broadly, the ``audience'' is often new to the topic.
Sometimes we are the \patternname{Newcomers}, sometimes we're the oldtimers.

\subsubsection*{Problem} \patternname{Newcomers} can feel overwhelmed by the amount of things to learn.  They
don't know where to start.\footnote{Peeragogy Project participant
R\'egis Barondeau: ``I joined this handbook project late, making me
a `newcomer'. When I started to catch up, I rapidly faced doubts:
Where do I start? How can I help? How will I make it, having to read
more than 700 posts to catch up? What tools are we using ? How do I use
them? Etc. Although this project is amazingly interesting, catching the
train while it already reached high speed can be an extreme sport. By
taking care of newcomers, we might avoid losing valuable contributors
because they don't know how and where to start, and keep our own project
on track.''}  They may have a bunch of ideas that the oldtimers have
never considered -- or they may think they have new ideas, which are actually
a different take on old ideas; see \patternname{Use or Make}.

\subsubsection*{Solution} It is good to try to become aware of what a \patternname{Newcomer}
needs, and what their motivations are.\footnote{Peeragogy Project participant
Charlotte Pierce: ``Joe was working a lot on the book, and I thought
`this is interesting hard work, and he shouldn't have to do
this alone.' As a Peeragogy newcomer, I was kindly welcomed and
mentored by Joe, Howard, Fabrizio, and others. I asked naive questions
and was met with patient answers, guiding questions, and resource links.
Concurrently, I bootstrapped myself into a position to contribute to the
workflow by editing the live manuscript for consistency, style, and continuity.''}
\patternname{Newcomers} themselves may have only a general idea about what their goals are, so it can be
helpful to add concreteness with \patternname{A Specific Project}.

\subsubsection*{Rationale} \patternname{Newcomers} in the Peeragogy project have often complained
about feeling confused about what the project is about, suggesting that our \patternname{Roadmap}
has not been sufficiently clear.  Some feel it is too theoretical, which suggests
we need to do more work on \patternname{Creating a guide} on ways to get involved, while also
making it clear that we do not have an exhaustive list in mind.  New ideas can prompt us to consider how we may have been limiting ourselves.\footnote{Dilrukshi Gamage, Julia Echeverria, and Federico Monaco first joined a Peeragogy hangout in February, 2015.  They were all interested in the idea of designing and running a course on peer learning.  Although we had done some work on a syllabus, and considered the notion of using peeragogical models in formal education, we hadn't tried running a course on the topic of peeragogy.  There were a number of earlier experiments that \emph{used} ideas from peeragogy inside of a formal course, and the difference between these two approaches prompted interesting discussions.}

\subsubsection*{Resolution}
The frustration and confusion felt by a \patternname{Newcomer} familiar to anyone who is starting something new.  An awareness of how to help \patternname{Newcomers} can help us be more compassionate to ourselves and others.

\subsubsection*{What's Next} A more detailed (but non-limiting) ``How to Get Involved'' walk-through in text or video form would be good to develop. We can start by listing some of the things we're learning about.\footnote{Business issues relevant to the Peeragogy project, how to run a MOOC, hot-syncing our website from Git, etc.}
