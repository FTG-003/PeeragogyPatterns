% Produce versus ???
% lingo is somewhat obscure: "Use or Make" - clarify?
\section{Use or make}\label{sec:Use_or_make}
\subsubsection*{Context}
% learning?
Peer production, as the name indicates, is about producing, in other words --
``making.'' But it also involves building on (``using'') the work of others.

\subsubsection*{Problem}
People are often very attached to their own projects and don't have a sense of how their own initiatives can benefit from connecting with others.

\subsubsection*{Solution} Learning actually involves recycling and remixing other ideas and techniques. Be mindful of the value of remixing!  And make it possible for other people to remix and adapt your work too.\footnote{As a first key step, we've released the \emph{Peeragogy Handbook} using the Creative Commons Public Domain Dedication (CC0).  This legal instrument grants the greatest possible leeway to downstream users; see \url{https://creativecommons.org/publicdomain/zero/1.0/}.  Contributors need to agree to the following terms: ``\emph{I hereby waive all copyright and related or neighboring rights together with all associated claims and causes of action with respect to this work to the extent possible under the law.}''  An email to one of the handbook editors or a comment to this effect on \url{http://peeragogy.org/resources/license/} suffices.}  Show appreciation when they do.  In the case of shared content, make backups so that you don't have to worry about losing the record of idea that the other person might not have noticed was important.

\subsubsection*{Rationale} 
Many projects die because the cost of \patternname{\href{http://c2.com/cgi/wiki?ReinventingTheWheel}{Reinventing the Wheel}} [c2] is too high.  \patternname{Creating a guide} can help people avoid reinventing the wheel.\footnote{Clearly we are not the first people to notice these things!  Consider the following quote from the Wikimedia Foundation: ``Unfortunately, many of our most valuable resources for learning and evaluation are scattered across wikis, buried in archived reports, incomplete, out of date, or are only available in a single language. As a result, we sometimes find ourselves re-inventing the wheel: missing opportunities, repeating common mistakes, and working harder than we need to because we are not aware of related projects done by others who came before us.'' via \url{https://blog.wikimedia.org/2013/11/19/learning-patterns-new-way-share-important-lessons/}.}  There are lots of tools out there -- \emph{use them}, at least on a trial basis; return to the ones that work.

\subsubsection*{Resolution} Noticing how difficult it is to remake things every time, and encapsulating what we observed with the \patternname{Use or Make} pattern reminds us to always consider re-purposing the work of others, to think about how others can leapfrog ahead, building on our experiences by incorporating our work.

\subsubsection*{What's Next} We've spun off the pattern catalog from the \emph{Peeragogy Handbook} into this paper, sharing it with a new community and gaining new perspectives.  Let's look for other parts of the handbook we can spin off!

