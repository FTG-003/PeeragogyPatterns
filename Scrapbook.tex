\section{Scrapbook}

\paragraph{Context:} We've maintained and revised our pattern catalog over a period of years.
\paragraph{Problem:} Some of the patterns don't seem to lead to concrete next steps.
\paragraph{Solution:} We've created a \emph{Scrapbook} for patterns that are no longer part of the active catalog.  It's worth remembering how we got to the point we're at now, and the thinking that developed along the way, so the \emph{Scrapbook} shows some patterns that seemed like a good idea at the time -- and brief notes about why we no longer need them explicitly.  We can use the \emph{Pattern Audit Routine} to decide when to move a pattern into the \emph{Scrapbook}.
\paragraph{Rationale:} We want our collection of patterns to be concretely useful and actively used.  It needs to be clear and pragmatic, and not overly theoretical or precriptive.  If we don't see ``what's next'' and where it came from, then it's probably time to shift focus to something else more practical.
\paragraph{Resolution:}  In revising our pattern catalog for PLoP we've decided to ``retire'' several patterns that seemed overly abstract or redundant (\emph{Discerning a pattern}, \emph{Polling for ideas}, \emph{Moderation}, \emph{Roles}) as well as some antipatterns that didn't suggest concrete next steps, and instead simply held a prism to our collective frustrations (\emph{Isolation}, \emph{Magical thinking}, \emph{Messy with Lurkers}, \emph{Misunderstanding power}, \emph{Navel Gazing}, \emph{Stasis}, \emph{Stuck at the level of weak ties}).  The current catalog is leaner and redescribes our project in an action-oriented way. 
\paragraph{What's next:} After significantly pruning back the pattern catalog, we want it to grow again: new patterns are needed.  Reviewing the contents of the \emph{Scrapbook} will be one place to look for inspiration, but there are others.

