% Produce versus ???
\section{Use or make}
\paragraph{Context:}
% learning?
Peer production, as the name indicates, is about producing, in other words --
``making.'' But it also involves building on the work of others.

\paragraph{Problem:}
People are often very attached to their own projects and don't have a sense of how their own initiatives can benefit from connecting with others.

\paragraph{Solution:} Learning actually involves recycling and remixing other ideas and techniques. Be mindful of the value of remixing!  And make it possible for other people to remix and adapt your work too.  Show appreciation when they do.  In a case of shared content, make backups so that you don't have to worry about losing an idea that the other person might not have considered.

\paragraph{Rationale:} 
Many projects die because the cost of reinventing the wheel is too high.  There are lots of tools out there -- consider and try different tools and return to the ones that work.

\paragraph{Resolution:} Noticing how difficult it is to remake things every time, and encapsulating what we observed with the \emph{Use or Make} pattern reminds us to always consider re-purposing the work of others, and has spurred us on to think about how others can leapfrog ahead, building on our experiences by incorporating our work.

\paragraph{What's Next:} We've spun off the pattern catalog from the \emph{Peeragogy Handbook} into this paper, sharing it with a new community and gaining new perspectives.  Let's look for other parts of the handbook we can spin off!

\paragraph{Cf.:} 
\begin{itemize}
\item Unfortunately, many of our most valuable resources for learning and evaluation are scattered across wikis, buried in archived reports, incomplete, out of date, or are only available in a single language. As a result, we sometimes find ourselves re-inventing the wheel: missing opportunities, repeating common mistakes, and working harder than we need to because we are not aware of related projects done by others who came before us. via https://blog.wikimedia.org/2013/11/19/learning-patterns-new-way-share-important-lessons/
\item  Reinventing The Wheel Go ahead. Grab that chisel and start pounding. You know you want to.  An idiom common amongst engineers and developers.  Definition 
\begin{enumerate}
\item Redundant effort
\item Inventing something that has already been invented, and that can't be (or hasn't been) improved upon
\item Unnecessarily performing a task that has already been successfully completely by others
\end{enumerate}          
Why invent a new wheel when you can walk to the store and buy one? Why invent a wheel when you can invent the engine? via http://c2.com/cgi/wiki?ReinventingTheWheel
\end{itemize}