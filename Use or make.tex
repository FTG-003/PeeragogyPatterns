% Produce versus ???
\section{Use or make}
\paragraph{Context:}
% learning?
Peer production, as the name indicates, is about producing, in other words --
``making.'' But it also involves building on the work of others.

\paragraph{Problem:}
People are often very attached to their own projects and don't have a sense of how the project can benefit from connecting with others.

\paragraph{Solution:} Learning actually involves recycling and remixing people's ideas and techniques. Be aware of the value of remixing!  And make it possible for other people to remix and adapt your work too.

\paragraph{Rationale:} 
Many project die because the cost of reinventing the wheel is too high.  We've tried using different tools and have stuck with the ones that work.

\paragraph{Resolution:} Noticing how difficult it is to remake things every time, and encapsulating what we observed with the \emph{Use or Make} pattern reminds us to always consider re-purposing the work of others, and has spurred us on to think about how others can leapfrog ahead, building on our experiences by adopting our work.

\paragraph{What's Next:} ``Platform'' debates can be frustrating but can
also add something to a project in the long term, since they help people
become aware of their priorities. As mentioned in
the \href{http://peeragogy.org/patterns/newcomer/}{Newcomer} pattern,
developing a more clear picture of the activities that we engage in in
the project will help make it comprehensible to others. It will also be
useful for us to have a clearer picture of what we do, and what we make.
