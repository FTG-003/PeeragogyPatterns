\section{User or make}
\paragraph{Definition:} Peer production, as the name indicates, is about
``making stuff.'' And making stuff can be fun and worthwhile. But we
should also ask ourselves, how much new stuff do we really need? Is
there something around that we could already use? There's not a hard and
fast answer to this question.

\paragraph{Problem:} Usually we end up working at both levels -- for
example, writing something new using an existing wiki, or creating a
piece of software that builds on someone else's API. Sometimes we have
to dig deeper, and recreate a system in a more bottom-up fashion. The
main issue at stake is to try to become clear about where you do and do
not need to start from scratch (and also, be aware of the fact that we
almost never really start from scratch).

\paragraph{Solution:} A lot of ``learning'' is really ``remix'' -- that is,
reuse and recycling of other people's ideas and techniques.
Understanding and negotiating the tension between reuse and creativity
is the key to the art of remix!

\paragraph{Challenges:} We've had interesting conversations recently about
the role of open source software in peeragogy. Most project participants
agree that the open source ideals are more important than strictly using
open source software for everything. Some feel that it would be best if
we create an open source alternative for any proprietary systems we use.
The debate has been an interesting and largely fruitful one: it's
mentioned here to point out that there's usually no one right answer to
``reuse'' questions.

\paragraph{What's Next:} ``Platform'' debates can be frustrating but can
also add something to a project in the long term, since they help people
become aware of their priorities. As mentioned in
the\href{http://peeragogy.org/patterns/newcomer/}{Newcomer} pattern,
developing a more clear picture of the activities that we engage in in
the project will help make it comprehensible to others. It will also be
useful for us to have a clearer picture of what we do, and what we make.

\begin{quote}
\textbf{Jean Baudrillard}: ``Praxis, a noble activity, is always one of
use, as distinct from poesis which designates fabrication. Only the
former, which plays and acts, but does not produce, is noble.'' {[}1{]}
(p. 101)
\end{quote}

\paragraph{Reference:}

\begin{enumerate}
\itemsep1pt\parskip0pt\parsep0pt
\item
  Baudrillard, J. (1975). The mirror of production. Telos Press
\end{enumerate}
