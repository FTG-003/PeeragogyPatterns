\section{Use or make}
\paragraph{Context:}
% learning?
Peer production, as the name indicates, is about producing, in other words --
``making stuff.'' But it also involves building on the work of others.

\paragraph{Problem:}
People often get very gung ho about their own projects and don't have a sense of how they can connect with others.

\paragraph{Solution:} A lot of ``learning'' is really ``remix'' -- that is,
reuse and recycling of other people's ideas and techniques.
Understanding and negotiating the tension between reuse and creativity
is the key to the art of remix!

\paragraph{Rationale:} We've had interesting conversations recently about
the role of open source software in peeragogy. Most project participants
agree that the open source ideals are more important than strictly using
open source software for everything. Some feel that it would be best if
we create an open source alternative for any proprietary systems we use.
The debate has been an interesting and largely fruitful one: it's
mentioned here to point out that there's usually no one right answer to
``reuse'' questions.

\paragraph{Resolution:} Writing this down reminds us to always consider repurposing things.

\paragraph{What's Next:} ``Platform'' debates can be frustrating but can
also add something to a project in the long term, since they help people
become aware of their priorities. As mentioned in
the \href{http://peeragogy.org/patterns/newcomer/}{Newcomer} pattern,
developing a more clear picture of the activities that we engage in in
the project will help make it comprehensible to others. It will also be
useful for us to have a clearer picture of what we do, and what we make.

\begin{quote}
\textbf{Jean Baudrillard}: ``Praxis, a noble activity, is always one of
use, as distinct from poesis which designates fabrication. Only the
former, which plays and acts, but does not produce, is noble.'' {[}1{]}
(p. 101)
\end{quote}

\paragraph{Reference:}

\begin{enumerate}
\itemsep1pt\parskip0pt\parsep0pt
\item
  Baudrillard, J. (1975). The mirror of production. Telos Press
\end{enumerate}
