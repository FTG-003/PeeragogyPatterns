\section{Navel Gazing}
\paragraph{Definition:} The difficulty breaks down like this:

\begin{enumerate}
\item
  Certainly we cannot get things done just by talking about them.
\item
  And yet, feedback can be useful, i.e., if there are mechanisms for
  responding to it in a useful fashion.
\item
  The associated anti-pattern is a special case of the prototypical
  Batesonian \href{http://en.wikipedia.org/wiki/Double_bind}{double
  bind}, ``the father who says to his son: go ahead, criticize me, but
  strongly hints that all effective criticism will be very unwelcome''
  {[}1{]}, p. 88.
\end{enumerate}

\paragraph{Problem:} Criticism is not always useful. Sometimes it is just
``noise''.

\paragraph{(Bogus) Solution:} It's tempting to create ``open'' systems that
inadvertantly replicate the double bind -- by being open to criticism,
but unable to act on it effectively.

\paragraph{Challenges:} A long list of criticisms that haven't been dealt
with is maybe better than no communication at all, but it's also a
tell-tale sign of deeper dissatisfaction. It's better to make sure you
have enough bandwidth (see
\href{http://peeragogy.org/patterns-usecases/patterns-and-heuristics/carrying-capacity/}{Carrying
Capacity}) for dealing with a given class of problems and issues. Adjust
your focus accordingly, but be careful (see
``\href{http://peeragogy.org/antipatterns/isolation/}{Isolation}'').

\paragraph{What's Next:} We have hinted that, in this project,
\href{http://peeragogy.org/how-to-use-this-handbook/}{effective
criticism is very welcome}! But understanding what makes criticism
effective is, in general, still a research problem.

\paragraph{Reference:}

\begin{enumerate}
\itemsep1pt\parskip0pt\parsep0pt
\item
  Deleuze, G., and Guattari, F. (2004). \emph{Anti-oedipus}. Continuum
  International Publishing Group.
\end{enumerate}
