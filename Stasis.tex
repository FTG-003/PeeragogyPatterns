\section{Stasis}
\paragraph{Definition:} Actually, living beings are never really in stasis.
It just sometimes feels that way. Other anti-patterns like
\href{http://peeragogy.org/antipatterns/isolation/}{Isolation} and
\href{http://peeragogy.org/antipatterns/navel-gazing/}{Navel Gazing}
have described different aspects of the experience of feeling like one
is in stasis. Typically, what is happening in such a case is that one or
more dimensions of life are moving very slowly.

\paragraph{Problem:} When important things are moving slowly or not at all,
and when they are mostly or entirely out of your control, this can be
frustrating.

\paragraph{(Bogus) Solution:} It's tempting in this case just to be upset and to
feel disempowered.

\paragraph{Example:} We were not able to get programming support to improve
the first version of the Social Media Classroom, since all developer
energy was allocated to the next version of the system. It becomes
frustrating if a specific small feature is desired, but unavailable.

\paragraph{Challenges:} Of course, it's very unpleasant to be frustrated
all the time. The hint to pick up is that there is always some dimension
on which you can make progress. It might not be the same one you've been
working on -- you might have ``over-harvested'' that niche (see
\href{http://peeragogy.org/patterns-usecases/patterns-and-heuristics/carrying-capacity/}{Carrying
Capacity}).

\paragraph{What's Next:} We're working on a new handbook chapter about the
relationship of open source software and peeragogy. This will include
some more specific ideas about ways of making change.
