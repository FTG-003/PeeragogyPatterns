\section{Introduction}

Our readers will certainly be familiar with Wikipedia, open source software, and StackExchange.  In the Peeragogy project, we are interested in understanding and exploring ``how that all works.''  We aim to build from these inspiring examples to the future of education.

There is a key difference between our pattern catalog and previous collections of design patterns that touch on similar domains -- like \emph{Liberating Voices: A Pattern Language for Communication Revolution} and \emph{Pedagogical Patterns: Advice for Educators}.  Our pattern catalog is that we are more action-oriented.  People usually use design patterns to talk about general solutions, whereas we have both general and specific-to-us aspects. The patterns evolve as the project evolves, and represent ``skin in the game.''

A quite convincing implementation of Alexander’s idea of patterns as a “living language” (Alexander et
al., 1977, p. xvii) was realized with one of the earliest applications of wiki
software developed by Ward Cunningham: the Portland Pattern Repository.
What we've developed is a further iteration of this idea. To use a visual metaphor, whereas other pattern languages are often top-down, ours is more horizontal.  Structure is more explicitly emergent.  These features are built
into our pattern template and the way we use the pattern catalog.  This is suitable for a project
that is itself emergent.

We are not currently focusing on programming, but we are using a lot of different pieces in order to working with a distributed teams of volunteers.  To give an idea: many of the patterns described in this paper first appeared in a private Drupal forum; they were then extracted and cached on a MediaWiki, published on a Wordpress blog, discussed via gMail, Google+, and Google Hangouts, ``hive edited'' in real time on Google Docs, typeset with XeLaTeX, hive edited again using ShareLaTeX, and then moved into Github with a bridge to Authorea for final edits.  Along the way, some of them were defended in a PhD viva, and the template was revised and revised again.
% and link to the videos and link to the source docs/videos

Between \emph{describing} and \emph{prescribing} there is a third pragmatic option: \emph{doing}.
The concrete offer to others who are thinking about patterns is an exploration of patterns-in-action
through peeragogy in action.