\section{Peeragogy Project}

\begin{quote}
This section introduces the \emph{Peeragogy Project} and the idea of \emph{design patterns} with an example.
\end{quote}

\textbf{Context:}  The Peeragogy project began in 2011.  To date it has been run by a small and changing group of interested volunteers, collaboring via real-time meetings and shared documents that describe the patterns and processes of peer learning we've observed in this project and others.   We are not currently an official organization, but we have some lofty goals.  As time goes by we would like to become be a knowledge capital competitor/collaborator with Wikipedia, somewhat akin to StackExchange.

Architectual maverick Christopher Alexander asked the following question to an audience of computer programmers in 1999: 
\begin{quote}
``What is the Chartres of programming? What task is at a high enough level to inspire people writing programs, to reach for the stars?''
\end{quote}
We believe that the nexus of learning and programming -- exemplified today by Wikipedia, StackExchange, and the Peeragogy Project -- may be just the thing.

\textbf{Problem:} In a volunteer context, telling people what to do really doesn't work.  So we need another way to communicate.

\textbf{Solution:} Design patterns and a distributed roadmap.
We think that Peeragogy and design patterns are a great match.  In fact, a ``Network of Learning'' was the 18th pattern that Christopher Alexander introduced in his book, A Pattern Language.  The network of learning we have in mind bridges physical and virtual worlds, moving from concrete to fantastical and back, fragmented by prism of different pespectives and unified by a desire to help make the world a better place. 

\textbf{Rationale:}
We can try to describe what we've learned, but the key thing is for everyone involved to keep learning. We can use our design pattern catalogue to build on what we've learned so far and increase opportunities for learning new thing.
Some relatively minor technical issues need to be resolved, and we
have to decide what platform to use

\textbf{Pattern:}
The idea of a shared \emph{Roadmap} has been with us since the beginning of the \emph{Peeragogy Project}.  Of course, the roadmap itself changes as time goes by.  The \emph{Roadmap}  exists as an artifact with which to share current, but never complete,  understanding of the space. Adding “What’s Next” steps to our patterns gives us a “distributed  roadmap.” And this works both ways: If we sense that something needs to  change about the project, that’s a clue that we might need to record a  new pattern.

\textbf{Resolution of forces:}  
This project has proved to be interesting, social, and with time may offer a range of new business model for education -- less lucrative perhaps, but more rewarding.  We've seen that participating in the peeragogy project gives us technical social and theory-building skills we can use in our day jobs for collaborating with colleagues.   Feel free to join us!