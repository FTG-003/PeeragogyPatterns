\section{Introduction}

\begin{itemize}
\item The key difference between other prior examples and our
patterns is we're making them explicitly action-oriented.  People usually use design patterns to talk about general
solutions, whereas we have both general and specific-to-us aspects.
The patterns evolve as the project evolves, and represent ``skin in the game.''
\item A quite convincing implementation of Alexander’s idea of patterns as a “living language” (Alexander et
al., 1977, p. xvii) was realized with one of the earliest applications of wiki
software developed by Ward Cunningham: the Portland Pattern Repository.
What we've developed is a further iteration of this idea.
\item To use a visual metaphor, whereas other pattern languages are often top-down,
ours is more horizontal.  Structure is more explicitly emergent.  These features are built
into our pattern template and the way we use the pattern catalog.  This is suitable for a project
that is itself emergent.
\item Our readers will certainly have heard of Wikipedia, open source
software, and StackExchange.  In the Peeragogy project, we are interested in
understanding and exploring ``how that all works.''
\item We are not focusing on software or programming aspects, but we
are really very good at working with distributed teams of volunteers and
at envisioning the future of education (and definitely we're using
computers).
\item Between \emph{describing} and \emph{prescribing} there is a third pragmatic option: \emph{doing}.
The concrete offer to others who are thinking about patterns is an exploration of patterns-in-action
through peeragogy in action.
\end{itemize}