\begin{itemize}
\item[{\bf Week of June 8}] Peeragogy Project and Roadmap, comments \#17--\#27.  JC: I've made an initial revision, changing ``Peeragogy Project'' to ``Peeragogy''.  See text in green below.  Roadmap is orange, meaning ``on deck'' but no major changes have been made there yet.
\item[{\bf Week of June 15}] Use or Make and Carrying Capacity
\item[{\bf Week of June 22}] A Specific Project and Wrapper
\item[{\bf Week of June 29}] Heartbeat and Creating a Guide
\item[{\bf Week of July 6}] Newcomer and Pattern Audit Routine
\item[{\bf Week of July 13}] Scrapbook and Distributed Roadmap
\item[{\bf Week of July 20}] Make sure to submit the revised paper
\end{itemize}

\section{Introduction}\label{sec:Introduction}

Readers will have encountered peer production, at least in applications like Wikipedia, StackExchange, and free/libre/open source software development.  In this paper we apply design patterns to understand the human side of this kind of socio-technical development.  In the Peeragogy project,  we aim to build upon these inspiring examples to help design the future of education.  

We have found design patterns tremendously useful for organizing our thinking about these matters.  However, there is a key difference between our pattern catalog and previous collections of design patterns that touch on similar domains -- like \emph{Liberating Voices: A Pattern Language for Communication Revolution} \cite{schuler2008liberating} and \emph{Pedagogical Patterns: Advice for Educators} \cite{bergin2012pedagogical}:

Our pattern catalog is our primary project management tool -- our way of ``synthesizing form''  \cite{alexander1964notes}.
A quite convincing implementation of Christopher Alexander’s idea of patterns as a ``living language'' \cite[p.~xvii]{alexander1977pattern} was realized with one of the earliest applications of wiki software developed by Ward Cunningham: the Portland Pattern Repository. What we've developed is a further iteration of this idea.  

% Many of the patterns described in this paper were first shared in a private Drupal forum, and were first made public on a Wordpress blog as part of our \emph{Peeragogy Handbook} \cite{peeragogy-handbook}.\footnote{\url{http://peeragogy.org}}  They were then discussed via gMail, Google+, and Google Hangouts, ``hive edited'' in real time on Google Docs, format-shifted with pandoc, typeset with XeLaTeX, hive edited again with ShareLaTeX, and then moved into Github and bridged to Authorea for final edits. 
As they developed, some of the patterns made an appearance at a PhD thesis defense \cite{corneli-thesis}, and the pattern template was revised and revised again, until ultimately we settled on something fairly traditional.  At the level of the template, one innovation is to add a ``What's next'' step to each pattern, which anticipates the way the pattern will continue to ``resolve'' in the context of our project. The result is a hands-on counterpart to existing sociological and historical research on peer production, surveyed in \cite{benkler2015peer}. 

At a more philosophical level, our approach is all about human interaction, and the challenges, fluidity and lack of predictability that comes with it.  Something that works for one person may not work for another or may not even work for the same person in a slightly different situation.  It is easy to say ``just do X'' and somewhat easy for reasonable people to agree in general terms about what to do.  In our view, other pattern languages frequently achieve this sort of common sense rationality, but then they stop.  In our experience, failure in the prescriptive model begins when people start trying to define things more carefully, make context-specific changes; when they actually try to put ideas into practice, or understand things in a more tangible way. The patterns we introduce here describe ways to negotiate the execution and implementation of solutions in their practical context.  This often requires compromise, adjustments and even restarts.

% Schematically, this is what the current template tells us:

% \begin{center}
% \begin{tikzpicture}[framed, dot/.style={circle,inner sep=1pt,fill,name=#1}]
\node (context) at (0, 0) {{ {\sc Context}}};
\node (problem) at (2.5, 0) {{ {\sc Problem}}};
\node (solution) at (5, 0) {{ {\sc Solution}}};
\node (rationale) at (7.65, 0) {{ {\sc Rationale}}};
\node (resolution) at (10.5, 0) {{ {\sc Resolution}}};
\node (whatnext) at (13.25, 0) {{ {\sc What next}}};

\node[below of=context] (where) {{ {\sc Where}}};
\node[below of=problem] (what)  {{ {\sc What}}};
\node[below of=solution] (how)  {{ {\sc How}}};
\node[below of=rationale] (why) {{ {\sc Why}}};
\node[below of=resolution] (who) {{ {\sc Who}}};
\node[below of=whatnext] (when)  {{ {\sc When}}};

\draw[-latex] (context) -- (where);
\draw[-latex] (problem) -- (what);
\draw[-latex] (solution) -- (how);
\draw[-latex] (rationale) -- (why);
\draw[-latex] (resolution) -- (who);
\draw[-latex] (whatnext) -- (when);
\end{tikzpicture}

% \end{center}

%% The authors are part of a shifting, globally distributed, team of volunteers with around 30 major contributors, and a long tail with over 1000 members.\footnote{\url{https://plus.google.com/communities/107386162349686249470}} 

Other aspects of the patterns -- as well as this idea of using design patterns as an organizational tool -- will be generally useful for students and educators who want their work to have real-world relevance, to activists and policy-makers who want to develop practicable solutions to large-scale problems, and to employees and managers who, like it or not, find themselves working in distributed teams.   Our approach to emergent organization will also be of interest to theorists of social interaction in fields like organization studies and, increasingly, computer science.  The next section introduces \patternname{Peeragogy} more explicitly in the form of a design pattern.  Sections \ref{sec:Roadmap}--\ref{sec:Scrapbook} list the main patterns in our catalog.    Figure \ref{fig:connections} illustrates their interconnections.  Section \ref{sec:Distributed_Roadmap} collects our ``What's next'' steps and summarizes the outlook of our project.

\begin{figure}
\vspace{-.9in}
{\centering
\begin{tikzpicture}[dot/.style={circle,inner sep=1pt,fill,name=#1}]
%\draw[step=1cm,gray,very thin] (0,0) grid (10,10);
\node (assess) at (5, 9.75) {{\Large {\sc Assess}}};
\node (organize) at (5, 0) {{\Large {\sc Organize}}};
\node (cooperate)[text width=2cm,align=center,rotate=270] at (10, 5) {{\Large {\sc Convene}}};
\node (convene)[text width=15cm,align=center,rotate=90] at (0, 5) {{\Large {\sc Cooperate}}};
%%
\node (roadmap) at (4.25,4.65) {\ref{sec:Roadmap}. \emph{Roadmap}};
%%
\node (useormake) at (3.15, 8.75) {\ref{sec:Use_or_make}. \hyperref[sec:Use_or_make]{\emph{Use or make?}}};
\node (par) at (6.85, 8.75) {\ref{sec:Pattern_Audit_Routine}. \hyperref[sec:Pattern_Audit_Routine]{\emph{Pattern Audit Routine}}};
%
\node (carryingcapacity) at (1.25, 7.15) {\ref{sec:Carrying_capacity}. \hyperref[sec:Carrying_capacity]{\emph{Carrying capacity}}};
\node (heartbeat) at (1.6, 4.5) {\ref{sec:Heartbeat}. \hyperref[sec:Heartbeat]{\emph{Heartbeat}}};
%
\node (aspecificproject) at (8.5, 6.5) {\ref{sec:A_specific_project}. \hyperref[sec:A_specific_project]{\emph{A specific project}}};
\node (creatingaguide) at (6.85, 2) {\ref{sec:Creating_a_guide}. \hyperref[sec:Creating_a_guide]{\emph{Creating a guide}}};
%
\node (wrapper) at (2.5, 2) {\ref{sec:Wrapper}. \hyperref[sec:Wrapper]{\emph{Wrapper}}};
\node (newcomer) at (8.5, 3.25) {\ref{sec:Newcomer}. \hyperref[sec:Newcomer]{\emph{Newcomer}}};
\node (scrapbook) at (5, 1) {\ref{sec:Scrapbook}. \hyperref[sec:Scrapbook]{\emph{Scrapbook}}};
%
\node[below=2cm of newcomer] (peeragogyproject) {\ref{sec:Peeragogy_Project}. \emph{Peeragogy Project}};
%
\draw[-{Latex[width=2mm]},draw=gray] (peeragogyproject) -- ([xshift=0pt]newcomer.south);
\draw[-{Latex[width=2mm]},draw=gray] (aspecificproject) -- (par);
\draw[-{Latex[width=2mm]},draw=gray] (aspecificproject) -- (roadmap);
\draw[-{Latex[width=2mm]},draw=gray] (carryingcapacity) -- (newcomer);
\draw[-{Latex[width=2mm]},draw=gray] (carryingcapacity) -- (roadmap);
\draw[-{Latex[width=2mm]},draw=gray] ([xshift=2mm]creatingaguide.160) to[out=-215,in=-67] (carryingcapacity);
\draw[-{Latex[width=2mm]},draw=gray] (heartbeat) -- (newcomer);
\draw[-{Latex[width=2mm]},draw=gray] (heartbeat) -- (scrapbook);
\draw[-{Latex[width=2mm]},draw=gray] (heartbeat) -- (useormake);
\draw[-{Latex[width=2mm]},draw=gray] (newcomer) -- ([xshift=4mm]useormake.south);
\draw[-{Latex[width=2mm]},draw=gray] (newcomer) -- (aspecificproject);
\draw[-{Latex[width=2mm]},draw=gray] (newcomer) -- (creatingaguide.north);
\draw[-{Latex[width=2mm]},draw=gray] (newcomer) -- (roadmap);
\draw[-{Latex[width=2mm]},draw=gray] (par) -- (scrapbook);
\draw[-{Latex[width=2mm]},draw=gray] (roadmap) -- (newcomer);
\draw[-{Latex[width=2mm]},draw=gray] (roadmap) -- (useormake);
\draw[-{Latex[width=2mm]},draw=gray] (scrapbook) -- (par);
\draw[-{Latex[width=2mm]},draw=gray] (scrapbook) -- (wrapper);
\draw[-{Latex[width=2mm]},draw=gray] ([xshift=2mm,yshift=-.4mm]useormake.south) -- (creatingaguide);
\draw[-{Latex[width=2mm]},draw=gray] (wrapper) -- (heartbeat);
\draw[-{Latex[width=2mm]},draw=gray] (wrapper) -- (roadmap);
\end{tikzpicture}


\par
}
\vspace{-.9in}
\caption{Connections between the patterns of peeragogy.  An arrow points from pattern \textbf{A} to pattern \textbf{B} if the description of pattern \textbf{A} references pattern \textbf{B}. Labels at the borders of the figure correspond to the main sections of the \emph{Peeragogy Handbook}.\label{fig:connections}}
\end{figure}


    