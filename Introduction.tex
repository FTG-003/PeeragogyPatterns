\section{Introduction}\label{sec:Introduction}

Readers will be familiar with Wikipedia, open source software, and StackExchange.  In the Peeragogy project, we are interested in understanding and exploring ``how that all works.''    We want to build upon these inspiring examples to help design the future of education.  Toward this end, we've developed peeragogy as a model for peer produced peer learning.

We have found design patterns tremendously useful for organizing our thinking about these matters.  However, there is a key difference between our pattern catalog and previous collections of design patterns that touch on similar domains -- like \emph{Liberating Voices: A Pattern Language for Communication Revolution} \cite{schuler2008liberating} and \emph{Pedagogical Patterns: Advice for Educators} \cite{bergin2012pedagogical}.  Our pattern catalog is more action-oriented, and evolves as the project evolves.   
% Between \emph{describing} and \emph{prescribing} there is a third pragmatic option, which is the one we tend to prefer: \emph{doing}.  

A quite convincing implementation of Alexander’s idea of patterns as a ``living language'' \cite[p.~xvii]{alexander1977pattern} was realized with one of the earliest applications of wiki software developed by Ward Cunningham: the Portland Pattern Repository. What we've developed is a further iteration of this idea. To use a visual metaphor, whereas other pattern languages are often top-down, ours is more horizontal.  Structure is emergent.  These features are built into our pattern template and the way we use the pattern catalog.  This is suitable for a project that is itself emergent.

While we are currently focusing on design, not programming, we do use a lot software in order to work together as a shifting, globally distributed, team of volunteers.  Many of the patterns described in this paper were first shared in a private Drupal forum, and were first made public on a Wordpress blog as part of our \emph{Peeragogy Handbook} \cite{peeragogy-handbook}.\footnote{\url{http://peeragogy.org}}  They were then discussed via gMail, Google+, and Google Hangouts, ``hive edited'' in real time on Google Docs, format shifted with pandoc, typeset with XeLaTeX, hive edited again with ShareLaTeX, and then moved into Github and bridged to Authorea for final edits.  Along the way, some of the patterns made an appearance at a PhD thesis defense \cite{corneli-thesis}, and the pattern template was revised and revised again.  
% and link to the videos and link to the source docs/videos

We believe this document will be useful for educators who want their teaching to have real-world relevance, and to activists who want to develop practicable solutions.   The approach to emergent order may also be of interest to theorists of social interaction in fields like organization studies and, increasingly, computer science.  The next section introduces the \emph{Peeragogy Project} in the form of a design pattern.  Sections \ref{sec:Roadmap}--\ref{sec:Scrapbook} list the main patterns in our catalog.  Section \ref{sec:Distributed_Roadmap} summarizes the outlook.  Figure \ref{fig:connections} shows connections between the patterns.

\begin{figure}
{\centering
\begin{tikzpicture}[dot/.style={circle,inner sep=1pt,fill,name=#1}]
%\draw[step=1cm,gray,very thin] (0,0) grid (10,10);
\node (assess) at (5, 9.75) {{\Large {\sc Assess}}};
\node (organize) at (5, 0) {{\Large {\sc Organize}}};
\node (cooperate)[text width=2cm,align=center,rotate=270] at (10, 5) {{\Large {\sc Convene}}};
\node (convene)[text width=15cm,align=center,rotate=90] at (0, 5) {{\Large {\sc Cooperate}}};
%%
\node (roadmap) at (4.25,4.65) {\ref{sec:Roadmap}. \emph{Roadmap}};
%%
\node (useormake) at (3.15, 8.75) {\ref{sec:Use_or_make}. \hyperref[sec:Use_or_make]{\emph{Use or make?}}};
\node (par) at (6.85, 8.75) {\ref{sec:Pattern_Audit_Routine}. \hyperref[sec:Pattern_Audit_Routine]{\emph{Pattern Audit Routine}}};
%
\node (carryingcapacity) at (1.25, 7.15) {\ref{sec:Carrying_capacity}. \hyperref[sec:Carrying_capacity]{\emph{Carrying capacity}}};
\node (heartbeat) at (1.6, 4.5) {\ref{sec:Heartbeat}. \hyperref[sec:Heartbeat]{\emph{Heartbeat}}};
%
\node (aspecificproject) at (8.5, 6.5) {\ref{sec:A_specific_project}. \hyperref[sec:A_specific_project]{\emph{A specific project}}};
\node (creatingaguide) at (6.85, 2) {\ref{sec:Creating_a_guide}. \hyperref[sec:Creating_a_guide]{\emph{Creating a guide}}};
%
\node (wrapper) at (2.5, 2) {\ref{sec:Wrapper}. \hyperref[sec:Wrapper]{\emph{Wrapper}}};
\node (newcomer) at (8.5, 3.25) {\ref{sec:Newcomer}. \hyperref[sec:Newcomer]{\emph{Newcomer}}};
\node (scrapbook) at (5, 1) {\ref{sec:Scrapbook}. \hyperref[sec:Scrapbook]{\emph{Scrapbook}}};
%
\node[below=2cm of newcomer] (peeragogyproject) {\ref{sec:Peeragogy_Project}. \emph{Peeragogy Project}};
%
\draw[-{Latex[width=2mm]},draw=gray] (peeragogyproject) -- ([xshift=0pt]newcomer.south);
\draw[-{Latex[width=2mm]},draw=gray] (aspecificproject) -- (par);
\draw[-{Latex[width=2mm]},draw=gray] (aspecificproject) -- (roadmap);
\draw[-{Latex[width=2mm]},draw=gray] (carryingcapacity) -- (newcomer);
\draw[-{Latex[width=2mm]},draw=gray] (carryingcapacity) -- (roadmap);
\draw[-{Latex[width=2mm]},draw=gray] ([xshift=2mm]creatingaguide.160) to[out=-215,in=-67] (carryingcapacity);
\draw[-{Latex[width=2mm]},draw=gray] (heartbeat) -- (newcomer);
\draw[-{Latex[width=2mm]},draw=gray] (heartbeat) -- (scrapbook);
\draw[-{Latex[width=2mm]},draw=gray] (heartbeat) -- (useormake);
\draw[-{Latex[width=2mm]},draw=gray] (newcomer) -- ([xshift=4mm]useormake.south);
\draw[-{Latex[width=2mm]},draw=gray] (newcomer) -- (aspecificproject);
\draw[-{Latex[width=2mm]},draw=gray] (newcomer) -- (creatingaguide.north);
\draw[-{Latex[width=2mm]},draw=gray] (newcomer) -- (roadmap);
\draw[-{Latex[width=2mm]},draw=gray] (par) -- (scrapbook);
\draw[-{Latex[width=2mm]},draw=gray] (roadmap) -- (newcomer);
\draw[-{Latex[width=2mm]},draw=gray] (roadmap) -- (useormake);
\draw[-{Latex[width=2mm]},draw=gray] (scrapbook) -- (par);
\draw[-{Latex[width=2mm]},draw=gray] (scrapbook) -- (wrapper);
\draw[-{Latex[width=2mm]},draw=gray] ([xshift=2mm,yshift=-.4mm]useormake.south) -- (creatingaguide);
\draw[-{Latex[width=2mm]},draw=gray] (wrapper) -- (heartbeat);
\draw[-{Latex[width=2mm]},draw=gray] (wrapper) -- (roadmap);
\end{tikzpicture}


\par
}
\vspace{-.9in}
\caption{Connections between the patterns described in this document (\emph{Roadmap} and \emph{Peeragogy Project} not shown).  Labels at the borders of the figure correspond to the main sections of the \emph{Peeragogy Handbook}.\label{fig:connections}}
\end{figure}
