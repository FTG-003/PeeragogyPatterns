\section{Peeragogy Project}

\begin{quote}
This section introduces the \emph{Peeragogy Project} in the form of a \emph{design pattern}.
\end{quote}

\textbf{Context:}  Architectual maverick Christopher Alexander asked the following questions to an audience of computer programmers in 1999: 
\begin{quote}
``What is the Chartres of programming? What task is at a high enough level to inspire people writing programs, to reach for the stars?''
\end{quote}
We believe that the nexus of learning and computers -- exemplified today by Wikipedia, StackExchange, and the \emph{Peeragogy Project} -- may be just the thing.

\textbf{Problem:} In a volunteer context, telling people what to do really doesn't work.  So we need another way to communicate.

\textbf{Solution:} Christopher Alexander introduced the idea of \emph{design patterns} -- using a simple template to describe and shape the human life world.  In our context, design patterns allow us to bridge physical and virtual, move from fantastic to concrete and back.  The template we use is relatively traditional.  We've made a few minor alterations to Alexander's original model, in order to use patterns in a project that is always changing.  Specifically, we think of each pattern as something active: we try to make it clear why documenting each pattern achieves, and we write down the ``What's next'' steps we have planned.  Like Alexander, we cross-reference our patterns to understand the links between them.  The \emph{Peeragogy Project} is itself an up-to-date example of one of Alexander's patterns, \href{http://en.wikipedia.org/wiki/Networked_learning#1970s}{\emph{Network of Learning}}.

\textbf{Rationale:}
Patterns are intuitive to write and read.  Our specific interperation of the framework helps us document what we've learned, and even moreso, helps everyone involved keep learning.  We use our design pattern catalogue to build on what we've learned so far, and to scaffold our work on other parts of the project -- our technical platform, our Handbook, our meetings -- and to connect in fruitful ways with other projects.  

\textbf{Pattern:}
The idea of a shared roadmap has been with us since the beginning of the \emph{Peeragogy Project}.  Of course, the roadmap itself changes as time goes by.  More precisely, the \emph{Peeragogy Project} uses actively updated patterns to develop a current but never complete distributed \emph{Roadmap} for our shared efforts.  If we sense that something needs to  change about the project, that’s a clue that we might need to record or adapt one of our patterns. 

\textbf{Resolution:}  
Writing down this pattern defines some of the key terms for \emph{Newcomers}, and illustrates the way our template works. 

\textbf{What's next:} 
With time, the \emph{Peeragogy Project} may suggest a range of new models for education.  Feel free to join us or adapt our pattterns to help shape another \emph{Peeragogy Project}.\footnote{The Peeragogy Project, without italics, refers to a specific example of the \emph{Peeragogy Project} pattern.   This project has been active since it was introduced by Howard Rheingold in 2011.  In order to enhance the readability of our patterns, examples, which often form part of the design pattern templates used by others, appear here in footnotes.}