\section{Stuck at the level of weak ties}
\paragraph{Definition:} Knowing how to make good use of ``weak ties'' is
often seen as a ``personal strength''.

\begin{quote}
\textbf{Nancy Darling}:  When we need a big favor or social or
instrumental support, we ask our friends. We call them when we need to
move a washing machine. But if we need information that we don't have,
the people to ask are our weak ties. They have more diverse knowledge
and more diverse ties than our close friends do. We ask them when we
want to know who to hire to install our washing machine. {[}1{]}
\end{quote}

The question is less to do with whether we are forming weak ties or
strong ties. We can be ``peers'' in either a weak or a strong sense. The
question to ask is whether our needs match our expectations!

\paragraph{Problem:} In the peeragogy context, this has to do with how we
interact.

\begin{quote}
\textbf{One of us}: I am learning about peeragogy, but I think I'm
failing to be a good peeragogue. I remember that Howard once told us
that the most important thing is that you should be responsible not only
for your own learning but for your peers' learning. {[}\ldots{}{]} So
the question is, are we learning from others by ourselves or are we
helping others to learn?
\end{quote}

If we are ``only'' co-consumers of information then this seems like a
classic example of a weak tie. We are part of the an ``audience''.

\paragraph{(Bogus) Solution:} Perhaps especially in an online, mediated, context,
it is possible to stay at the level of ``weak ties.''
(Cf. \href{http://peeragogy.org/antipatterns/isolation/}{Isolation}.)

\paragraph{Challenges:} This strategy reveals its problems directly, if you
ever need help moving your washing machine.

\paragraph{What's Next:} If we are actively engaging with other people,
then this is a foundation for strong ties. In this case of deep
learning, our aims are neither instrumental nor informational, but
``interactional''. Incidentally, the ``One of us'' quoted above has been
one of the most consistently engaged peeragogues over the years of the
project. Showing up is a good step -- you can always help someone else
move their washing machine!

\paragraph{Reference:}

Nancy Darling (2010).
\href{http://www.psychologytoday.com/blog/thinking-about-kids/201005/facebook-and-the-strength-weak-ties}{Facebook
and the Strength of Weak Ties}, \emph{Psychology Today}.
