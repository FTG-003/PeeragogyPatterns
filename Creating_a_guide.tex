\section{Creating a guide}\label{sec:Creating_a_guide}
\subsubsection*{Definition} Meaning-carrying tools, like handbooks or maps, can
help people use an idea, collecting content and stories.

\subsubsection*{Problem} When the idea or system is only ``newly discovered'',
the associated meanings may not be well understood, and indeed they may
not have been created. Even if a topic is only ``personally new'', it
can be hard to find ones way around.

\subsubsection*{Solution} In such a case, the process of creating the guide can
go hand-in-hand with figuring out how the system works. Thus, techniques
of \href{http://knowledgecartography.org/}{knowledge cartography}
and \href{http://www.hitl.washington.edu/publications/r-97-47/two.html}{meaning
making} are useful for would-be guide creators.\footnote{We started the Peeragogy project by collaboratively
making an outline for the Peeragogy Handbook. We recommended this
handbook-making practice to others, as a way to learn collaboratively
and build a strong group.}

\subsubsection*{Challenges} Remember that ``the map is not the territory,'' and
map-making is only one facet of shared human activity. For instance, a
pattern description can be thought of as a ``micro-map'' of a specific
activity. These maps are not useful if they are divorced from practice.

\subsubsection*{What's Next} We've been talking with collaborators in the
Commons Abundance Network about how to make a Pattern Language for the
Commons. One of the challenges that arises is how to support ongoing
development of the Pattern Language itself: a ``living'' map for a
living territory. We're refining the Peeragogy Pattern language and
template as a seed for this.


