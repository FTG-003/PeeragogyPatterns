\section{Template}

\begin{quote}
This section describes our ``adapted'' template, and currently contains a discussion of a new contender.  We'll have to decide which one to use.
\end{quote}

For consistency, and easy use, adaptation, and extension we present
the patterns using the following template.~ The format is meant to be
neutral and easy to work with -- it's, intentionally, an outline that
you might use to write a short abstract describing an active project.

\begin{quote}
\textbf{Title}: \emph{Encapsulate the idea - possibly include a
subtitle}

\textbf{Definition}: \emph{Explain the idea and the context in which it
is meaningful. ~(You can link to other patterns, if they are useful for
clarifying the relevant context.)}

\textbf{Problem}: \emph{Explain why there's some issue to address here.}

\textbf{Solution}: \emph{Talk about an idea about how to address the
issue.}

\textbf{Challenges}: \emph{Talk about what can go wrong.}

\textbf{Examples}: \emph{Added as needed to illustrate the other points.}

\textbf{What's Next}: \emph{Talk about specific next steps. ~(Again,
link to other patterns, if they are useful for clarifying the relevant
context.)}
\end{quote}

Notice the emphasis the active aspect of things -- the ``What's Next''
section concretely links the patterns we discuss here to the Peeragogy
project.~ If you adapt them for use in your own project, you're likely
to have a different set of next steps. Although we think that these
patterns can be generally useful, they aren't useful in the abstract,
but rather, as a way for discussing what we actually do.

\emph{NOTE:} I think we should discuss switching to a slightly
different template.

\begin{quote}
\textbf{Context:} What we formerly called ``Definition''

\textbf{Problem:} As above.

\textbf{Solution:} As above.

\textbf{Rationale:} Why we chose this solution.

\textbf{Resolution:} What problems we've solved \emph{by writing down the pattern}.

\textbf{What's Next:} What other problems remain to be solved.
\end{quote}
