\section{Template}

For consistency, and easy use, adaptation, and extension we present
the patterns using the following template.~ The format is meant to be
neutral and easy to work with -- it's, intentionally, an outline that
you might use to write a short abstract describing an active project.

\begin{quote}
\textbf{Title}: \emph{Encapsulate the idea - possibly include a
subtitle}

\textbf{Definition}: \emph{Explain the idea and the context in which it
is meaningful. ~(You can link to other patterns, if they are useful for
clarifying the relevant context.)}

\textbf{Problem}: \emph{Explain why there's some issue to address here.}

\textbf{Solution}: \emph{Talk about an idea about how to address the
issue.}

\textbf{Challenges}: \emph{Talk about what can go wrong.}

\textbf{What's Next}: \emph{Talk about specific next steps. ~(Again,
link to other patterns, if they are useful for clarifying the relevant
context.)}

The pattern template also includes the following optional elements:

{[}\textbf{Objectives}: \emph{Explain the purpose(s) of the proposed
solution's functioning, if they aren't fully specified by the
description of the solution itself.}{]}

{[}\textbf{Examples}: \emph{Present example(s) that have been
encountered, if this aids comprehension.}{]}

{[}\textbf{References}: \emph{Citations, if relevant.}{]}
\end{quote}

Notice the emphasis the active aspect of things -- the ``What's Next''
section concretely links the patterns we discuss here to the Peeragogy
project.~ If you adapt them for use in your own project, you're likely
to have a different set of next steps. Although we think that these
patterns can be generally useful, they aren't useful in the abstract,
but rather, as a way for discussing what we actually do.

