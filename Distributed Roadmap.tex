\section{Distributed Roadmap}

\begin{quote}
This section summarizes the ``What's Next'' steps in all the previous
patterns to give an overview of our current ``distributed roadmap''
for the Peeragogy project.
\end{quote}

\paragraph{Carrying Capacity:} This pattern catalog has been rewritten in a way that should make it
easy for anyone to add new patterns. Making it easy and fruitful for
others to get involved is one of the best ways to redistribute the load
(compare the Newcomer pattern).

\paragraph{Creating A Guide:} We’ve been talking with collaborators in the Commons Abundance Network
about how to make a Pattern Language for the Commons. One of the
challenges that arises is how to support ongoing development of the
Pattern Language itself: a “living” map for a living territory. We’re
refining the Peeragogy Pattern language and template as a seed for this.

\paragraph{Discerning A Pattern:} What do the patterns we’ve observed say about the self-selection
processes of the group? For instance, it’s possible that a widespread
interest in organic gardening, say, may indicate the participants are
oriented to cooperation, personal health, or environmental activism.
What can we learn about the Peeragogy project from our collected
patterns?

\paragraph{Heartbeat:} When the project is bigger than more than just a few people, it’s likely
you’ll get several heartbeats – for instance, we’ve recently been
running two weekly meetings in the Peeragogy project, for members with
slightly different interests and slightly different availability.
Finding ways to communicate across these different “camps” is useful.

\paragraph{Newcomer:} We recently revised the “How to Get Involved” page, listing the top ten
sites we use. Another reasonable thing to post would be a top-ten list
of activities, so that people can get an easier view on the kinds of
things we do in the project.

\paragraph{Polling For Ideas:} We’ve considered asking new members of the project to do an “entry survey” as part of joining the project, to describe their aims and
understanding of what they hope to contribute. This could establish a
context of contribution, and help new members to feel like full “peers”.

\paragraph{Use or Make:} “Platform” debates can be frustrating but can also add something to a project in the long term, since they help people become aware of their
priorities. As mentioned in the Newcomer pattern, developing a more clear
picture of the activities that we engage in in the project will help
make it comprehensible to others. It will also be useful for us to have
a clearer picture of what we do, and what we make.

\paragraph{Roadmap:} Adding ``What's Next'' steps to our patterns gives us a ``distributed roadmap.''  And this works both ways:  
If we sense that something needs to change about the project, that's a
clue that we might need to record a new pattern.

\paragraph{Roles:} We’ve listed some of the roles for which we’re seeking volunteers in
the Peeragogy.org Roadmap: Volunteer Coordinator, Seminar Coordinator,
Usability Guru, Activities Master, and Tech lead. As with everything
else in the roadmap, this list should be reviewed and revised
regularly, as the roles are understood relative to the actual
happenings in the Peeragogy project.

\paragraph{Specific Project:} In the third year of the Peeragogy project, rather than just keep
working on the handbook, we’ve been working on building a Peeragogy
Accelerator, as a peer support system for projects related to peer
learning and peer production. Not only does specificity help member
projects, being clear about what the Accelerator itself is supposed to
do will help people get involved.

\paragraph{Wrapper:}  We need better practices for wrapping things up at
various levels.  One of the latest ideas is to develop a simple visual
``dashboard'' for the project.
