\section{Roadmap}

\subsubsection*{Definition:} It is very useful to have an up-to-date public
roadmap for the project, a place where it can be discussed and
maintained. The Roadmap exists as an artifact with which to share
current, but never complete, understanding of the space.

\subsubsection*{Problem:} Without a roadmap, there will not be a shared sense of
the project's goals or working methods. It will be much harder for
people to volunteer to help out, or to assess the project's progress.

\subsubsection*{Solution:} Keeping a list of current and upcoming activities, as
well as goals and working methods can
help \href{http://peeragogy.org/practice/heuristics/newcomer/}{newcomers}
and old-timers alike see where they can jump in. As we cross things off
the list, this gives a sense of the accomplishments to date, and any
major challenges that lie ahead.\footnote{In the Peeragogy project, once the handbook's outline became fairly
  mature, we could use it as a roadmap, by marking the sections that are
  ``finished'', marking the sections where editing is currently taking
  place, and marking the stubs (possible starting points for future
  contributors). After this outline matured into a
  real \href{http://peeragogy.org/table-of-contents/}{table of contents},
  we started to look in other directions for things to work on, and
  created a \href{http://peeragogy.org/peeragogy-org-roadmap/}{roadmap
  for further development of the website and peeragogy project as a
  whole}}\textsuperscript{,}\footnote{There can be a certain roadmappiness to ``presentation of self'', and
  you can learn to use this well. For instance, when introducing
  yourself and your work to other people, you can focus on highlights
  like these: ``\emph{What is the message behind what you're doing?}''
    ``\emph{How do you provide a model others can follow or improve upon?}''
    ``\emph{How can others get directly involved with your project?}''}

\subsubsection*{Challenges:} Unless the roadmap is easy for people to see and to
update, they are not likely to use it. In the Peeragogy Accelerator
phase of the project, we've included a roadmap in the ``behind the
scenes'' version of our landing page, we're using it as a way to link to
other documents we're working on. Accordingly, people participating in
the accelerator frequently encounter the roadmap as a ``first level''
object. All of this said, sometimes it's impossible to know in advance
what will happen! A roadmap that's not quite right will feel burdening.
Sometimes it's better to become more open to the unknown.

\subsubsection*{What's Next:} 
Adding ``What's Next'' steps to our patterns gives us a ``distributed
roadmap.''  And this works both ways: If we sense that something needs
to change about the project, that's a clue that we might need to
record a new pattern. https://docs.google.com/document/d/1RZEsqFDwF-jPiCvgWzJgi6n6faTRTDuPQS1CMEeXxRE/edit\#heading=h.p197njr3jsn8 http://www.metis.progedit.com/anno-iii-numero-1-giugno-2013-formare-tra-scienza-tecnica-tecnologia-temi/99-saggi/490-roadmaps-in-peer-learning.html
