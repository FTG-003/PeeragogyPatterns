\begingroup \color{OliveGreen}

\section{Roadmap} \label{sec:Roadmap}

% DK: This is a bit self-referential. You have roadmap embedded throughout the description of the pattern. E.g. the context and problem should be able to describe the situation before the solution has been applied, so you should be able to describe them without the solution name in them.

% DK: How is this different from a Backlog? Who “owns” the roadmap? How are changes made to it? How precise is it? Does it have a time dimension to it? You refer to deadlines later on. Does the roadmap include them? (What do deadlines really mean in a project like this anyhow?) Priority?

\subsubsection*{Context} As discussed above, \patternname{Peeragogy} has both distributed and centralized aspects. In simplest terms, the several different discussants or contributors have different points of view and differing priorities, but they come together to share these in conversations and joint activities.

\subsubsection*{Problem} In order to collaborate, people need a way to share current, though incomplete, understanding of the space they are working in, and the nature of our relationships with one another and the other elemetns of this space.  Without a sense of personal goals, outstanding problems, and working methods, there is no way for people to volunteer to help out, or even to assess the project's progress.  

\subsubsection*{Solution} Keeping a list of current and upcoming activities, as well as goals and working methods can help \patternname{\href{http://peeragogy.org/practice/heuristics/newcomer/}{Newcomers}} and old-timers alike see where they can jump in.  This can take many forms, and can have different levels of detail.  The solution may take the form of an outline of a draft document, a backlog of issues in an issue tracker, a succinct organizational mission statement, a course syllabus, a manifesto, or a calendar of upcoming events.  The distinguishing feature of a roadmap in the peeragogical sense is that it must be adaptive to circumstances.  The roadmap should be accessible to everyone with an interest in the project, though in practice not everyone will choose to update it.  One of the jobs of the project's \patternname{Wrapper} is to help synthesize an accurate roadmap in lieu of widespread participation, or in case of conflict.

\subsubsection*{Rationale} Unless the project's plan is easy for people to see and to update, they are not likely to use it, and are less likely to get involved.  The roadmap can also impose other limitations, for example it may be ``owned'' exclusively by a central committee -- however the key point of the roadmap is to help support involvement by those who \emph{are} involved.  The opportunity for feedback (if this feedback is taken seriously) enhances a project's peeragogical aspects.  The level of detail in the roadmap (and the existence of a roadmap at all) should correspond to the felt need for sharing information and to the tolerance of uncertainty among participants.
% DK: This seems more like advice about how to implement the solution than it is an explanation of how the solution addresses the forces from the context/problem [jc: fixed]

\subsubsection*{Resolution}
% DK: This seems like Rationale to me [jc: fixed]
The roadmap can shift along with its contents: in the Peeragogy project it has ranged from an outline of the first draft of the Peeragogy Handbook to a calendar of meetings with a regular ``\patternname{Heartbeat}'' that has helped to sustain the project. Simply describing a list of nice-to-have features is not likely to \emph{go} anywhere: this shows the difference between a mere backlog of tasks and a realistic plan for getting from here, to there.  Only now with our latest strategy for using the pattern language as an organizational tool do we have a robust mechanism in place for building and maintaining a ``distributed roadmap.''

\begin{framed}
\emph{What's Next.}
If we sense that something needs to change about the project, that is a clue that we might need to record a new pattern.
\end{framed}

\endgroup
    
    
    