\section{Peeragogy Project}

\begin{quote}
This section introduces the \emph{Peeragogy Project} in the form of a \emph{design pattern}.
\end{quote}

\paragraph{Context:}  Architectual maverick Christopher Alexander asked the following questions to an audience of computer programmers in 1999: 
\begin{quote}
``What is the Chartres of programming? What task is at a high enough level to inspire people writing programs, to reach for the stars?''
\end{quote}
We believe that the nexus of learning and computers -- exemplified today by Wikipedia, StackExchange, and Free/Libre/Open Source Software (FLOSS) -- presents an implicit challenge to the old way of doing things in education.  Taking up that challenge and building a new model is reaching \emph{ad astra per aspera}.  

\paragraph{Problem:} In a volunteer context, telling people what to do really doesn't work.  So we need another way to communicate.  Furthermore, everyone involved in these projects seems to be learning all the time.  So the way we communicate needs to be adaptive to circumstances.

\paragraph{Solution:} Christopher Alexander introduced the idea of \emph{design patterns} -- using a simple template to describe and shape the human life world.  In our context, design patterns allow us to bridge physical and virtual, move from fantastic to concrete and back.  The template we use is relatively traditional.  We've made a few minor alterations to Alexander's original model, in order to use patterns in a project that is always changing.  Specifically, we think of each pattern as something active: in addition to the traditional \emph{context}/\emph{problem}/\emph{solution} and \emph{rationale}, we try to make it clear what the very act of documenting each pattern \emph{resolves}, and we record the ``\emph{What's next}'' steps we have planned, in order to make the currently acting forces explicit. As time goes by we revise.  Like Alexander, we cross-reference our patterns to understand the links between them. The \emph{Peeragogy Project} is itself an up-to-date example of one of Alexander's patterns, \href{http://en.wikipedia.org/wiki/Networked_learning#1970s}{\emph{Network of Learning}} [APL].

\paragraph{Rationale:}
Patterns are intuitive to write and read. Our specific interpretation of the framework helps everyone involved integrate what they are learning.  We can use our design pattern catalog to scaffold our work on other parts of the project -- our technical platform, our Handbook, our meetings -- and to connect in fruitful ways with other projects.  

\paragraph{Resolution:}  
Writing down this pattern defines some of the key terms for \emph{Newcomers}, and illustrates the way our template works. 

\paragraph{What's next:} 
Feel free to join us and suggest new patterns and projects, or adapt our patterns to help shape another \emph{Peeragogy Project}.\footnote{In the present document, the term Peeragogy Project, without italics, refers to a current historical, real-world, example of the \emph{Peeragogy Project} pattern. This project has been active since it was introduced by Howard Rheingold in 2011. In order to enhance the readability of our patterns for a general audience, \emph{examples} drawn from our experiences in the Peeragogy Project and related projects, which would typically be part of a design pattern template, appear in footnotes.}