\section{Carrying capacity}\label{sec:Carrying_capacity}
\subsubsection*{Definition} There's only so much any one person can do in a
project.

\subsubsection*{Problem} At times, a facilitator or participant in the
peer-learning enterprise may feel he or she is over-contributing -- or,
perhaps more likely, that others are under-contributing -- or that
someone else is railroading an idea or dominating the discussion.

\subsubsection*{Solution} If this happens, take a step back and observe the
dynamics of involvement. Ask questions and let others answer. Especially
if you start to feel the symptoms of burnout, it's important that you
find the level of engagement that allows you to participate at a level
that is feasible for maintaining progress toward the project's goal.
Lead by example -- but make sure it's someplace you, and others,
actually want to go! This could be a good time to revisit the group's
roadmap and see if you can figure out and clarify to others what
concrete goal you're working towards. Remember that you can also change
the ``landscape'' by making it easier for other people to get involved
-- for example, by explaining what you're trying to do in a clear
manner. Be on the look out for opportunities to step back, watch, and
listen. Try to be mindful of phases when active or quiet involvement
would be more helpful to the individual and the group. It's also helpful
to let anyone who has taken on a facilitation role know if you're
stepping back temporarily. Then, when the time is right, step back in
and get to work!

\subsubsection*{Challenges} Even though your project may be very important, you
won't always make it go better by working harder.

\begin{quote}
\textbf{Alvin Toffler}: If overstimulation at the sensory level
increases the distortion with which we perceive reality, cognitive
overstimulation interferes with our ability to `think.'
\end{quote}

If you notice yourself caring about the outcomes more than other
participants, investigate why this is. Are you all affected by the
outcomes in the same way? Working smart requires you to focus on your
goals, while relating to others who may have a different outlook, with
different, but still compatible goals.

\subsubsection*{What's Next} This pattern catalog has been rewritten in a way
that should make it easy for anyone to add new patterns. Making it easy
and fruitful for others to get involved is one of the best ways to
redistribute the load (compare
the \href{http://peeragogy.org/practice/heuristics/newcomer/}{Newcomer}
pattern).

