\section{A specific project}
\paragraph{Context:}
You find yourself interested in or concerned about something.

\paragraph{Problem:}
It's easy to think about issues that matter: there are many of them. The problem is figuring out what you're going to do about it.

\paragraph{Solution:} 
Being concrete about what you'd like to do, learn, and achieve, takes you from thinking about a topic to becoming a practitioner.  You may realize that your ``specific project'' is too large to tackle directly. In this case, you will have to become even more specific.  Maintaining a project \emph{Roadmap} can help keep track of the smaller pieces and the bigger picture.

\paragraph{Rationale:} 
Specificity is relatively important in order for things to change.\footnote{In the January, 2013, plenary
session, \href{http://ipne.org}{Independent Publishers of New England}
(IPNE) President Tordis Isselhardt quietly listened to a presentation
about how we created the \emph{Peeragogy Handbook}. During the Q\&A, she
spoke up, wondering if peer-learning effort in IPNE might be more likely
to succeed if the organization's members ``focused around a specific
project.'' As this lightbulb illuminated the room, those of us attending
the plenary session suggested that IPNE could focus the project by
creating an ``Independent Publishing Handbook.'' (Applause!) In the
course of creating the IPNE Handbook, peer learners would assemble
resource repositories, exchange expertise, and collaboratively edit
documents. To provide motivation and incentive to participate in
``PeerPubU'', members of the association will earn authorship credit for
contributing articles, editor credit for working on the manuscript, and
can spin off their own chapters as stand-alone, profit-making
publications.} But while actions speak louder than words, it's important
to act in a coherent way if you want to be understood by others.  However, in
general it would be a mistake to try to seek consensus before acting.

\paragraph{What's Next:}  Each project connected with the \emph{Peeragogy Project} should be described with one or more patterns with specific, tangible ``what's next'' steps.  If the what's next step is not concrete enough, the \emph{Pattern Audit Routine} will help.  
