\section{A specific project}
\paragraph{Definition:} Being concrete about what you'd like to do, learn,
and achieve, takes you from thinking about a topic to becoming a
practitioner.

\paragraph{Problem:} It's easy to think about issues that matter: there are
many of them. The problem is figuring out what you're going to do about
it.

\paragraph{Solution:} Specificity is relatively important in order for
things to happen. Values -- and even metrics -- tend to be less
concretely meaningful than acts. At the same time, while actions speak
louder than words, it's important to act in a coherent way if you want
to be understood by others.

\paragraph{Example:} In the January, 2013, plenary
session, \href{http://ipne.org}{Independent Publishers of New England}
(IPNE) President Tordis Isselhardt quietly listened to a presentation
about how we created the \emph{Peeragogy Handbook}. During the Q\&A, she
spoke up, wondering if peer-learning effort in IPNE might be more likely
to succeed if the organization's members ``focused around a specific
project.'' As this lightbulb illuminated the room, those of us attending
the plenary session suggested that IPNE could focus the project by
creating an ``Independent Publishing Handbook.'' (Applause!) In the
course of creating the IPNE Handbook, peer learners would assemble
resource repositories, exchange expertise, and collaboratively edit
documents. To provide motivation and incentive to participate in
``PeerPubU'', members of the association will earn authorship credit for
contributing articles, editor credit for working on the manuscript, and
can spin off their own chapters as stand-alone, profit-making
publications.

\paragraph{Challenges:} As often happens, you may realize that your
specific goal is great, as a goal, but too large to tackle directly. It
this case, you may have to find a smaller piece of the project to focus
on. There will, eventually, be the problem of putting together the
little pieces in a coherent way.

\paragraph{What's Next:} In the third year of the Peeragogy project, rather
than just keep working on the handbook, we've been working on building a
Peeragogy Accelerator, as a peer support system for projects related to
peer learning and peer production. Not only does specificity help member
projects, being clear about what the Accelerator itself is supposed to
do will help people get involved.
