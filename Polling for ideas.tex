\section{Polling for ideas}
\paragraph{Definition:} Polling for Ideas can happen at many junctures in a
peer learning experience. We could poll for ideas like ``what's
missing?'', ``who might like to join our group?'', and ``what are the
right tools and resources for us to use at this point?''

\paragraph{Problem:} We recognize that we don't always know the answers in
advance -- particularly if we're trying to come up with an answer that
satisfies other people.

\paragraph{Solution:} Others might have an important or useful idea. Even
if all you can supply is the question and a context for discussion, they
may be willing to contribute these answers.\footnote{Near the start of the project, Howard suggested that
we use the forum categories he set up, but opened the floor to other
ideas, in this way: ``At the beginning, until we all know the ropes well
enough to understand when to create a new discussion forum topic and
when to add to an existing one, let's talk in this topic thread about
what else we want to discuss and I will start new topic threads when
necessary.''}

\paragraph{Challenges:} In the Peeragogy in Action Google+ community, and
in our earlier Social Media Classroom forum, both of which have been
open to any suggestion at all, we've many different ideas appear and
then roll by without being catalogued. When you have too much data, it
can be necessary to take separate steps to organize it. At the other end
of the spectrum, people don't always respond to surveys, and you can end
up with less data than you'd like.

\paragraph{What's Next:} We've considered asking new members of the project
to do an ``entry survey'' as part of joining the project, to describe
their aims and understanding of what they hope to contribute. This could
establish a context of contribution, and help new members to feel like
full ``peers''.
