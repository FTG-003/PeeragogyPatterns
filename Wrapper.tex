\section{Wrapper}\label{sec:Wrapper}
\subsubsection*{The Definition} A project peeragogue who summarizes what has happened recently in the project, making progress comprehensible to participants who have not been following all of the details.

\subsubsection*{Problem} Joining a project already going can feel like trying to get aboard a rapidly moving vehicle. If you joined and then taken time off, you may feel like things have moved on so far that they cannot catch up. In a very active project, it can be effectively impossible to stay up to date with all of the details.

\subsubsection*{Solution}
Adopting an idea from his Indiana University class led by Faridah Pawan Charlie Danoff suggested that someone take on the ``wrapper role'' -- do a weekly pre/post wrap, so that new (and existing) users would know the status the project is at any given point in time. The project's \href{http://socialmediaclassroom.com/host/peeragogy/}{landing page} also serves as another sort of ``wrapper'', telling people what they can expect to find.

\subsubsection*{Rationale}
The wrapper must check the public summaries of the project from time to time to make sure that they accurately represent the facts on the ground.\footnote{In the first year of the Peeragogy project, the ``Weekly Roundup'' by Christopher Tillman Neal served to engage and re-engage members. Peeragogues began to eager watched for the weekly reports to see if our teams or our names had been mentioned. When there was a holiday or break, Chris would announce the hiatus, to keep the flow going. In the second year of the project, we did not routinely publish summaries of progress, and instead, we've assumed that interested parties will stay tuned on Google+. It tutned out that was not enough, so Charlie has begun publishing irregular wrap-up blog posts and e-mails to keep people up to date. Regular meetings also provide an alternative or additional way to stay up to date: see the \href{http://peeragogy.org/patterns/heartbeat/}{Heartbeat} pattern.}

\subsubsection*{Resolution} 
According to the theory proposed by Yochai Benkler, for free/open ``commons-based'' projects to work, it is vital to have both (1) the ability to contribute small pieces; (2) something that stitches those pieces together \cite{coases-penguin}. The wrapper helps perform this integrative stitching function, which is often much more challenging than the job of breaking things down into pieces or just doing one of the small pieces.

\subsubsection*{What's Next}
We need better practices for automating the wrapping process at various levels. One of the latest ideas is to develop a simple visual ``dashboard'' for the project that would be a web page people could access and immediately get an idea of what work is ongoing in the project with the option to click links to go more in depth and/or contribute themselves.


