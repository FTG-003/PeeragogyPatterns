\paragraph{The Definition:} The wrapper role can be taken on by a project
participant who summarizes everything going on in the project, making
the project comprehensible to participants who haven't been following
all of the details.

\paragraph{The Problem:} Joining the project that is already going can feel
like trying to get aboard a rapidly moving vehicle. If you've joined and
then taken time off, you may feel like things have moved on so far that
it's impossible to catch up. In a very active project, it can be
effectively impossible to stay up to date with all of the details.

\paragraph{The Solution:} Charlie
Danoff \href{http://socialmediaclassroom.com/host/peeragogy/wiki/rolesdivision-labor}{suggested}
that someone take on the ``wrapper role'' -- do a weekly pre/post wrap,
so that new (and existing) users would know the status the project is at
any given point in time. The
project's \href{http://socialmediaclassroom.com/host/peeragogy/}{landing
page} also serves as another sort of ``wrapper'', telling people what
they can expect to find.

\paragraph{Objectives:} In fulfilling the wrapper role, we must check the
public summaries of the project from time to time to make sure that they
accurately represent the facts on the ground.

\paragraph{Examples:} In the first year of the Peeragogy project, the
``Weekly Roundup'' by Christopher Tillman Neal served to engage and
re-engage members. Peeragogues began to eager watched for the weekly
reports to see if our teams or our names had been mentioned. When there
was a holiday or break, Chris would announce the hiatus, to keep the
flow going. In the second year of the project, we didn't routinely
publish summaries of progress, and instead, we've assumed that
interested parties will stay tuned on Google+. Nevertheless, we maintain
internal and external summaries, ranging from agendas to press releases
to quick-start guides. Regular meetings provide an alternative way to
stay up to date: see
the \href{http://peeragogy.org/patterns/heartbeat/}{Heartbeat} pattern.

\paragraph{Challenges:} According to the theory proposed by Yochai Benkler,
for free/open ``commons-based'' projects to work, it is vital to have
both (1) the ability to contribute small pieces; (2) something that
stitches those pieces together {[}1{]}. The wrapper performs this
integrative function, which is often much more challenging than the job
of breaking things down into pieces or just doing one of the small
pieces.

\paragraph{What's Next:}
We need better practices for wrapping things up at
various levels.  One of the latest ideas is to develop a simple visual
``dashboard'' for the project.

\paragraph{Reference:}

\begin{enumerate}
\itemsep1pt\parskip0pt\parsep0pt
\item
  Benkler, Y.
  (2002). \href{http://www.yale.edu/yalelj/112/BenklerWEB.pdf}{Coase's
  Penguin, or Linux and the Nature of the Firm}, Yale Law Journal 112,
  pp. 369-446.
\end{enumerate}
