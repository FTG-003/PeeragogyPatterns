\section{Heartbeat}

\paragraph{Definition:} The project's heartbeat is a recurring activity,
something that makes it so that people experience a ``there, there.''

\paragraph{Problem:} Without someone or something acting as the heartbeat
for the group, energy may dissipate.

\paragraph{Solution:} People seem to gravitate to regularly scheduled
activities. Sometimes people need a little extra prompt to join in.

\paragraph{Examples:} In the ``Collaborative Lesson Planning'' course led
by Charlie Danoff at P2PU, Charlie wrote individual emails to people who
were signed up for the course and who had disappeared, or lurked but
didn't participate. This kept a healthy number of the people in the
group to re-engage and make positive contributions. In more recent
months, Charlotte Pierce has been running weekly meetings by Google
Hangout to coordinate work on the Peeragogy Handbook. Not only have we
gotten a lot of hands-on editorial work done this way, we've generated a
tremendous amount of new material (both text and video footage) that is
likely to find its way into future versions of the book.

\paragraph{Challenges:} Meetings that happen for the sake of having a
meeting are almost a bad joke. Be aware of the energy that's there
before and after meetings. If the energy isn't sustaining you or your
group, think about what needs to change.

\paragraph{What's Next:} When the project is bigger than more than just a
few people, it's likely you'll get several heartbeats -- for instance,
we've recently been running two weekly meetings in the Peeragogy
project, for members with slightly different interests and slightly
different availability. Finding ways to communicate across these
different ``camps'' is useful.
