\section{Heartbeat}\label{sec:Heartbeat}

\subsubsection*{Context}
People have a shared interest, and have connected with each other about it.

\subsubsection*{Problem} What's an easy way for these people feel like there's a ``there, there?''

\subsubsection*{Solution} People seem to naturally gravitate to regularly scheduled
activities. Once a week (meetings) or once a year (conferences, festivals) are two common variants.  Sometimes people need a little extra prompt to join in.\footnote{In the ``Collaborative Lesson Planning'' course led
by Charlie Danoff at P2PU, Charlie wrote individual emails to people who
were signed up for the course and who had disappeared, or lurked but
didn't participate. This helped course participants
re-engage and make positive contributions. In more recent
months, Charlotte Pierce has been the primary driver behind weekly meetings run using Google
Hangout to coordinate work on the \emph{Peeragogy Handbook} as well as tending to to other community interests. Not only have we
gotten a lot of hands-on editorial work done this way, we've generated a
tremendous amount of new material (both text and video footage) that is
likely to find its way into future versions of the \emph{Peeragogy Handbook}.  As long as this continues, ``Mondays at 1PM Eastern US time'' is a reliable time to make contact with the Peeragogy Project.}

\subsubsection*{Rationale}  This pattern might seem too obvious, since regularly scheduled meetings are so ubiquitous.  But there's an important difference between a mere meeting and a \patternname{Heartbeat}: in short, if the energy from your meetings isn't helping you or your group thrive, something needs to change.

\subsubsection*{Resolution} This pattern is one of the easiest ones to use to introduce \patternname{Newcomers} to the idea of a design pattern, since everyone is familiar with the pattern of a supportive routine.  But it is also a sophisticated tool: noticing when a new \patternname{Heartbeat} is beginning to emerge is a way to be aware of the priorities in the group, and may be a good source of new patterns.  Similarly, noticing when a specific \patternname{Heartbeat} has faded may be a sign that one of our patterns should be moved to the \patternname{Scrapbook}.

\subsubsection*{What's Next} When the project is bigger than more than just a few people, it's likely to have several \patternname{Heartbeats}.\footnote{There have been several periods of time operated two regularly scheduled weekly meetings in the Peeragogy project at distinct time slots, for members with slightly different interests and slightly different availability.  This typically relates to small special-purpose projects, like our work on this paper.}  Identifying and fostering new \patternname{Heartbeats} and new working groups is a task that can help make the community more robust.  This is the temporal dimension of spin off projects described in \patternname{Use or Make}.

