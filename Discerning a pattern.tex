\section{Discerning a pattern}
\paragraph{The Definition:} Discerning patterns helps us build our
vocabulary or repertoire for peer-learning projects. (The classic
example of an architectural pattern is ``A place to wait'' -- a type of
space found in many architectural and urban design projects.)

\paragraph{The Problem:} We might notice an underlying pattern if something
repeats, and if we're paying attention. However, unless we make a record
of the patterns we notice, others cannot and will not learn from our
experience, and with time, we'll forget what we learned.

\paragraph{The Solution:} Writing down patterns achieves at least two
things: it helps us pay attention and notice patterns in the first
place, and it provides a concrete summary of collective experience that
is relatively easy for others to engage with and extend. Once a pattern
is detected, give it a title and write down how the pattern works.

\paragraph{Challenges:} People may not be in the habit of writing down
patterns that they observe, and they are not likely to do it if the task
is not made easy and painless. Some projects that use the design pattern
methodology have developed detailed templates to gather information, but
this then needs to be processed by experts. We've tried to use a simple
template that is not much different from what you'd find in any short
textual abstract, to help make it easy to contribute new patterns.
Understanding how a given pattern relates to other patterns already
listed in in the catalog -- or to the wider context -- is not something
that can be easily encapsulated with templates. But it is still well
worth trying to express.

\paragraph{What's Next:} What do the patterns we've observed say about the
self-selection processes of the group? For instance, it's possible that
a widespread interest in organic gardening, say, may indicate the
participants are oriented to cooperation, personal health, or
environmental activism. What can we learn about the Peeragogy project
from our collected patterns?
